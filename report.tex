\documentclass{ituthesis}
\usepackage{hyperref} 

\settitle{The Practical Guide to Levitation}
\setauthor{Ahmad Salim Al-Sibahi}
\setsupervisor{Dr. Peter Sestoft}
\setextrasupervisor{David R. Christiansen}
\setdate{September 1, 2014}

\begin{document}
%\selectlanguage{danish}

\frontmatter

\thetitlepage
\newpage

\chapter*{Abstract}
Goal: Implementation of levitation in a realistic setting, with practical performance benefits.

\cleardoublepage
\setcounter{tocdepth}{1}
\tableofcontents

\mainmatter

%from memoir documentation:
%TeX tries very hard to keep text lines justified while keeping the interword spacing as constant as possible, but sometimes fails and complains about an overfull hbox.
%The default mode for LaTeX typesetting is \fussy where the (variation of) interword spacing in justified text is kept to a minimum. Following the \sloppy declaration there may be a much looser setting of justified text.
%Additionally the class provides the \midsloppy declaration which allows a setting somewhere between \fussy and \sloppy.
%fewer overfull lines than \fussy, and fewer obvious large interword spacing than with \sloppy.
%the memoir manual also uses \midsloppy!
\midsloppy
% try harder to avoid widows and orphans
\sloppybottom

\chapter{Prologue}
\label{cha:Prologue}
\section{Introduction}
\label{sec:Introduction}
\section{Problem Definition}
\label{sec:ProblemDefinition}
\chapter{Generic Programming}
\label{cha:GenericProgramming}
\section{The Generic Structure of Inductive Data Types}
\label{sec:TheGenericStructureofInductiveDataTypes}
\textit{How Generic Programming generally works.}
\section{The Importance of Genericity in Dependently-typed Languages}
\label{sec:TheImportanceofGenericityinDependently-typedLanguages}
\textit{The similarity of structure and various slightly-different indexing of types.}
\section{The (Mostly) Gentle Art of Levitation}
\label{sec:TheMostlyGentleArtofLevitation}
\textit{The elegance of a complete theorem for both ordinary and generic programming. Highlighting of possible issues with performance.}
\chapter{Partial Evaluation}
\label{cha:PartialEvaluation}
\section{Functions and Constant Inputs}
\label{sec:FunctionsandConstantInputs}
\textit{General introduction about partial evaluation.}
\section{Binding-time Analyses of Programs}
\label{sec:Binding-timeAnalysisofPrograms}
\textit{Finding the relevant constant parts of the program.}
\section{Specialisation as a Form of Optimization}
\label{sec:SpecialisationasaFormofOptimization}
\textit{Performance benefits of program specialisation. Pitfalls.}
\chapter{Levitating Idris}
\label{cha:LevitatingIdris}
\section{A Pragmatic Implementation of Levitation}
\label{sec:APragmaticImplementationofLevitation}
\textit{How the general concept of levitation was transferred to Idris.}
\section{Data Type Synthesis from Descriptions}
\label{sec:DataTypeSynthesisfromGenericDescriptions}
\textit{How levitated descriptions get transformed to ordinary data-types.}
\section{Static Initialization of Generic Functions}
\label{sec:StaticInitializationofGenericFunctions}
\textit{How algorithms that are dependent on the generic structure of a data-type are optimized. Discuss benefits of having a JIT/Profiling information for future work.}
\chapter{Practical Examples}
\label{cha:PracticalExamples}
\section{Generic Deriving}
\label{sec:GenericDeriving}
\textit{Examples of generic deriving of algorithms like decidable equality, pretty printing and possibly eliminators via generic structure.}
\section{Uniplate for Idris}
\label{sec:UniplateforIdris}
\textit{A version of the Uniplate library for Idris based on} \url{http://community.haskell.org/~ndm/uniplate/} \textit{and} \url{http://www-ps.informatik.uni-kiel.de/~sebf/projects/traversal.html} \textit{.
This is useful for traversing structures in a generic fashion and especially when dealing with small changes in large data structures (such as compiler ADTs)}
\chapter{Epilogue}
\label{cha:Epilogue}
\section{Evaluation}
\label{sec:Evaluation}
\section{Future Work}
\label{sec:FutureWork}
\section{Conclusion}
\label{sec:Conclusion}

\end{document}
